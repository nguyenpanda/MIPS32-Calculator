%
%   Conclusion
%
%   What did we find?
%   
\clearpage
\section{Conclusion}
In general, we have conducted several test models and received many positive outcomes that provide strong proof to reject the null hypothesis that there is no significant relationship between these categories, therefore confirming our prediction that the thermal power design of each CPU strongly depends on the number of cores and base frequency, as well as the temperature that can occur when entering hard mode of task of the devices it is attached on, leaving chip producer a clearer view to balance out between the power consumption and the performance of the chips to better their benefits in the technology field.

With the topic \textbf{Researching on the Thermal Design Power of CPU} uses the R programming language to process statistical data and realize the model linear regression model, our team had a more intuitive view of how to extract data, process and analyze raw data, turning them into valuable data sources long-term, or better yet, being able to generalize the general situation and make predictions about the data set.

During the process of implementing the thesis will not be able to avoid some minor errors, as well as some of the predictions might not be exactly as reality, our group hopes that this thesis report will provide another aspect for the viewers about the issue of dealing with the Thermal Design Power problem in the real world.