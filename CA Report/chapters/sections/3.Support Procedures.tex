\renewcommand{\arraystretch}{1.5}

\section{Description of Support Procedures}

\label{sec:3.SupportProcedures}

All procedures follow the convention of using \$a0-\$a3 as arguments and \$v0-\$v1 as return registers, taken from registers with smaller to larger values. All registers used by the procedures are stored in main memory, so when encountering the \texttt{jr \$ra} instruction, all registers are restored to their values before the subprocedure calculation.

For functions that need to use registers in Coprocessor 1 (\texttt{\$f<x>}), here is my convention for register management:

\begin{itemize}
    \label{sec:3.Coprocessor Management}
    \item \verb|Return|\space\space\space\space: \$f0 - \$f3
    \item \verb|Temporary|: \$f4 - \$f11
    \item \verb|Argument|\space\space: \$f12 - \$f15
    \item \verb|Temporary|: \$f16 - \$f19
    \item \verb|Constant|\space\space: \$f24 - \$f31
\end{itemize}

All privates prefixed with double underscores are indications of private procedures, only to be used within their own procedures or when called. The coder should not actively call them out.

\subsection{Description of character and string handling}
    The procedures below aid two main procedures, \texttt{INFIX\_TO\_POSTFIX} [\ref{sec:5.INFIX_TO_POSTFIX}] and \texttt{EVALUATE\_POSTFIX} [\ref{sec:5.EVALUATE_POSTFIX}], \texttt{STACK} [\ref{sec:4.Stack}], as well as other related procedures. They include character checks and string-handling functions
    
    \begin{longtable}{|m{2.5cm}|m{2.5cm}|m{2.5cm}|m{1cm}|m{5.6cm}|}
        \hline
            \textbf{Procedure} & 
            \textbf{Args} & 
            \textbf{Return} & 
            \centering \textbf{Leaf fucnt} &
            \textbf{Description}\\
        \hline
        \endfirsthead
            IS\_ OPERAND& 
            char = \$a0&
            bool = \$v0&
            \centering X&
            Checking if a character is an operand.\\
        \hline
            IS\_ OPERATOR& 
            char = \$a0&
            bool = \$v0&
            \centering X&
            Checking if a character is in (+, -, *, /, \^{}, !).\\
        \hline
            OPERATOR\_ PRECEDENCE& 
            char = \$a0&
            bool = \$v0&
            \centering X&
            Getting the precedence of an operator.\\
        \hline
            STRING\_ LENGTH& 
            str = \$a0&
            int = \$v0&
            \centering X&
            Getting the length of a string by looping through the string until it encounters the null or newline character. This procedure has a time complexity of \(O(n)\).\\
        \hline
            COMPARE\_ STRING& 
            str = \$a0
            \newline str = \$a1&
            bool = \$v0&
            &
            Comparing two strings by looping through both strings, return true if both strings are the same length and have identical contents. This operation has a time complexity of \(O(n)\).\\
        \hline
            RESET\_ STRING& 
            str = \$a0
            \newline str = \$a0&
            \centering void&
            \centering X&
            Resetting the space containing the string by looping through and replacing each character with a null character. This operation has a time complexity of \(O(n)\).\\
        \hline
        \caption{Description of Support Procedures} \\
    \end{longtable}

\subsection{Mathematics procedures}
    \begin{longtable}{|m{2.5cm}|m{2.5cm}|m{2.5cm}|m{1cm}|m{5.6cm}|}
        \hline
            \textbf{Procedure} & 
            \textbf{Args} & 
            \textbf{Return} & 
            \centering \textbf{Leaf fucnt} &
            \textbf{Description}\\
        \hline
        \endfirsthead
            FACTORIAL& 
            int = \$a0&
            int = \$v0&
            &
            Calculating the factorial of an integer using recursion. If the input is greater than 0 and less than 16, raise an exception and end the program. This operation has a time complexity of \(O(n)\).\\
        \hline
            EXPONENT& 
            double = \$f12
            \newline double = \$f14&
            double = \$f0&
            \centering X&
            Calculating the exponent of \$f12 \^{} \$f14 by looping. It automatically floors \$f14 into an integer and can calculate if \$f14 < 0.  This operation has a time complexity of \(O(|int(\$f14)|)\). Notice that if \$f12 = \$f14 = 0, raise an exception\\
        \hline
        \caption{Description of Mathematics Procedures} \\
    \end{longtable}

\subsection{Utility procedures}
    In this subsection, there will be utility procedures that do not contribute significantly to the algorithms of the \texttt{INFIX\_TO\_POSTFIX} [\ref{sec:5.INFIX_TO_POSTFIX}] and \texttt{EVALUATE\_POSTFIX} [\ref{sec:5.EVALUATE_POSTFIX}] procedures. They are used solely for display and assignment requirements.

    \begin{longtable}{|m{3.3cm}|m{2.9cm}|m{1.3cm}|m{1.1cm}|m{5.6cm}|}
        \hline
            \textbf{Procedure} & 
            \textbf{Args} & 
            \textbf{Return} & 
            \centering \textbf{Leaf fucnt} &
            \textbf{Description}\\
        \hline
        \endfirsthead
        \hline
            \textbf{Procedure} & 
            \textbf{Args} & 
            \textbf{Return} & 
            \centering \textbf{Leaf fucnt} &
            \textbf{Description}\\
        \hline
        \endhead
            END\_PROGRAM& 
            \centering -&
            void&
            \centering X&
            The exit prompt will be displayed, and the system call \$v0 = 10 will be invoked to end the program.\\
        \hline
            TYPE\_QUIT& 
            \centering -&
            void&
            \centering -&
            The quit prompt will be displayed, and the \texttt{END\_PROGRAM} procedure will be called.\\
        \hline
            WRITE\_ TO\_FILE& 
            message = \$a0 \newline filename = \$a1 \newline mode = \$a2&
            void&
            \centering -&
            This procedure is used to \emph{write} only with create (\$a2=1) or \emph{append} (\$a2=9) strings to a file. It automatically measures the length of the string to write to the file by calling the \texttt{STRING\_LENGTH} procedure and closes the file after writing to it.\\
        \hline
            new\_line& 
            \centering -&
            void&
            \centering X&
            When printing newline characters to the screen, note that this function will be convenient for code writers but will significantly slow down the program due to unnecessary procedures.\\
        \hline
        \caption{Description of Utility Procedures} \\
    \end{longtable}

    \newpage
    
    \subsubsection{Print Procedures}
        \label{sec:3.Print}
        The procedures below are considered leaf procedures supporting screen printing by calling \texttt{syscall}.
        
        \begin{multicols}{2}
            \begin{itemize}
                \label{sec:3.Print function}
                \item \verb|PRINT_DOUBLE|: double = \$f12
                \item \verb|PRINT_STRING|: string = \$a0
                \item \verb|PRINT_CHAR|\space\space\space: char = \$a0
                \item \verb|PRINT_INT|\space\space\space\space: int = \$a0
            \end{itemize}
        \end{multicols}

        Example:
        \begin{code}{ASM}
            # Integer
            li $a0, 2024
            jal PRINT_INT

            # Character
            li $a0, 97
            jal PRINT_CHAR

            # Double (The index of Coprocessor must be even)
            mov.d $f12, $f8 # Assume $f8 = 09052024.1259
            jal PRINT_DOUBLE
        \end{code}
        \begin{lstlisting}[language=ASM, caption=Color procedures]
        \end{lstlisting}
    
    \subsubsection{Color Procedures}
        \label{sec:3.Color}
        All procedures below are leaf procedures, simply using the \texttt{jal <COLOR LABEL>} command before and after a print command will suffice.
    
        \begin{multicols}{4}
            \begin{itemize}
                \item \verb|RED|
                \item \verb|GREEN|
                \item \verb|YELLOW|
                \item \verb|BLUE|
                \item \verb|MAGENTA|
                \item \verb|CYAN|
                \item \verb|RESET|
            \end{itemize}
        \end{multicols}
    
        Example:
        \begin{code}{ASM}
            jal CYAN
            la $a0, ascii_string # ascii_string: .asciiz "I'm studying CA at BKU\n"
            jal PRINT_STRING
            jal RESET
        \end{code}
        \begin{lstlisting}[language=ASM, caption=Color procedures]
        \end{lstlisting}
                